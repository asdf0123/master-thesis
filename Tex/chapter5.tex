\chapter{总结与展望}\label{chap:conclusion}{
	\section{本文工作总结}
	对双界面智能卡物理安全性的分析和评估是密码工程领域研究的一项重要课题。抵御侧信道分析的能力是密码硬件单元物理安全性的一个重要方面。为了保障分析评估结果的准确性和可靠性,本文开展了对双界面智能卡的侧信道分析方法的研究。基于深度学习侧信道分析(DL-SCA)方法具有为智能卡物理安全性分析测评实践提供新颖解决方案和精选先进技术的巨大潜能。然而,使用DL-SCA对双界面商用智能卡ECDSA运行过程中采集的的电磁迹进行侧信道分析时,遇到了样本数量不足、单个样本过大的问题,严重地影响了这类方法的实际攻击效果。本文从\jiaodu 出发,提出了一种基于深度学习的电磁分析方法,并在一类双界面商用智能卡ECDSA实现进行了实证研究。
	
	本文的主要研究成果如下:
	
	(1)提出了面向基于深度学习侧信道分析的通用数据增强方法,在侧信道领域权威公开数据集上进行了实验验证。该方法核心思想是主控制器依据反馈信号自适应地调整数据增强策略参数。通过面向基于深度学习侧信道分析的通用数据增强方法,可以减少DL-SCA\chenggongtiaoshu。在AES\_HD、DPA v4、ASCADf(N=0/N=50/N=100)三种数据集五个场景下,与未采用数据增强方法的DL-SCA效果相比,\chenggongtiaoshu 分别降低21\%、25\%、24\%、48\%以及5\%。在AES\_HD、DPA v4、ASCADf(N=0/N=50)三种数据集四个场景中,与采用其他数据增强方法的DL-SCA效果相比,\chenggongtiaoshu 分别降低22\%、40\%、10\%以及30\%。实验结果表明了本文提出的数据增强方法的有效性,这为后续对双界面商用智能卡ECDSA实现的电磁子迹进行数据增强,减少电磁迹采集时间提供了保障。
	%本文从数据增强的角度,研究如何提升DL-SCA技术效果。本文提出一种通用数据增强方法,该方法能够自适应地搜索数据增强策略参数,增广出新的样本用于训练深度学习模型,进而提升DL-SCA效果。该方法主要由采用模拟退火机制的主控制器、采用组合式数据增强机制的数据增强单元以及提供侧信道攻击代价反馈信号的攻击评估单元三个核心组件构成,其核心思想是采用模拟退火机制依据反馈信号自适应地选择数据增强策略参数。本文提出的通用数据增强方法能够显著提升多种密码实现的 DL-SCA 效果。与未采用数据增强方法的DL-SCA效果相比,在AES\_HD、DPA v4、ASCADf(N=0/N=50/N=100)三种数据集五个场景下,\chenggongtiaoshu 分别降低21\%、25\%、24\%、48\%以及5\%。与采用其他数据增强方法的DL-SCA效果相比,在AES\_HD、DPA v4、ASCADf(N=0/N=50)三种数据集四个场景下,\chenggongtiaoshu 分别降低22\%、40\%、10\%以及30\%。
	
	
	(2)提出了一种面向ECDSA实现的基于深度学习的电磁分析方法,在一款双界面商用智能卡ECDSA实现上进行了实验验证,并给出了相应参考实现。该方法包括预处理、模板构建以及模板匹配三个阶段。通过基于深度学习的电磁分析方法,可以恢复ECDSA一次性随机数部分比特,并在与格方法结合时恢复ECDSA私钥。
	预处理阶段采用了\poifanwei 和\yuchuli 两项技术,首先运用了\poifanwei 技术将特征点范围从5000万缩小到1.8万,接着使用\yuchuli 技术提取了特征点,完成了电磁迹的预处理,成功构造了用于深度学习的数据集。
	模板构建阶段采用了\shujuzengqiang,选择最优的数据增强策略参数对现有数据集中的训练集进行数据增强,使得实际攻击时在不额外采集电磁迹的情况下将DL-SCA的\zyx 由91.35\%提高到95.50\%。
	模版匹配阶段采用了\jiashejianyanguji ,利用格方法攻击目标双界面商用智能卡ECDSA实现的特点(对错误预测为比特0敏感,错误预测为比特1不敏感),使用更严格的标准估计比特0。在五次实验中,在训练完成后样本数量为5000的测试集中平均有981个敏感中间值估计值为0,其准确率为100\%;有4019个敏感中间值估计值为1,其准确率为75.89\%。\zyx 达到了100\%。实际攻击时,利用被恢复的比特泄漏可以过滤出119个含有连续4比特0的nonce,达到了使用由nonce部分比特恢复私钥的格方法的条件。
	%本文以型号为J3D081的双界面商用智能卡ECDSA实现为例,开展DL-SCA相关研究,应用\chapref{chap:search1}所提方法,有效提升了DL-SCA在双界面商用智能卡分析场景中的技术效果。%在缺少双界面商用智能卡产品设计细节、ECDSA实现所使用的防护对策未知的情况下,成功实现了ECDSA签名私钥的破解。
	%本文提出并使用\yuchuli 预处理技术,以增加3天的预处理时间为代价,将DL-SCA时间由315天减小为8分钟,将运用格方法的时间由约$2\times10^{85}$年减小到约$7\times10^{6}$年。本文使用\chapref{chap:search1}的面向基于深度学习侧信道分析的通用数据增强方法,将运用格方法的时间由约$7\times10^{6}$年减小到约12.5年。本文提出并使用\jiashejianyanguji 方案,将运用格方法的时间由约12.5年减小到400秒。在缺少双界面商用智能卡产品设计细节的情况下,成功实现了ECDSA签名私钥的破解。
	本文的分析思路和技术方法可以应用于其他类型的智能卡分析场景中,为开展其他类似密码硬件单元的分析与破解提供可行的思路和真实的案例。
	
	
	\section{未来工作展望}
	本文主要从\jiaodu,提升了DL-SCA技术效果,在多个AES实现公开数据集以及一个ECDSA实现私有数据集上完成了验证。数据增强方法具有进一步研究的价值:
	\begin{itemize}
		\item 预处理和数据增强之间的转化关系有待探究。预处理和数据增强都是对能量迹或电磁迹进行某种变换,提高深度学习侧信道攻击技术效果,但是它们的变换方式有可能完全相反。例如,数据增强向训练数据添加噪声,但是预处理力求减小噪声提高信噪比。数据增强只用于深度学习的训练数据,用于攻击的能量迹不会进行数据增强。只有用于攻击的能量迹才会进行预处理操作。探究预处理和数据增强之间的转化关系有助于同时利用预处理和数据增强的优势,突破只使用数据增强或只使用预处理带来的技术瓶颈,进一步提升DL-SCA的实用性。
		\item 本文提出数据增强方法中的数据增强单元具有良好的扩展性,其中的数据增强子策略可以根据不同的侧信道攻击需求进行替换或增加,以适应更复杂算法实现运行过程中采集的数据集。本文中因为所选用的三种数据增强子策略几乎正交,所以顺序可以互换\footnote{仅交换执行子策略的顺序可以通过调整子策略参数完成。}。当选择更多数据增强子策略时,子策略之间的耦合关系对DL-SCA技术效果的影响也值得探究。
		\item 本文提出的通用数据增强方法,在理论上适用于不同的深度学习模型。数据增强旨在增广训练集,此过程与其他直接优化深度神经网络的技术并不冲突。因此,可以同时应用通用数据增强方法和深度神经网络优化技术来提升DL-SCA技术效果,但是本文未进行相应实验进行验证。
	\end{itemize}
}