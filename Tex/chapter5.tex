\chapter{总结与展望}\label{chap:conclusion}{
	\section{本文工作总结}
	双界面商用智能卡因为其灵活性和可靠性成为智能化社会中各种商用安全应用的重要技术,因此对双界面智能卡物理安全性的分析和评估具有重要的现实意义。抵御侧信道分析的能力是密码硬件单元物理安全性的一个重要方面。为了保障分析评估结果的准确性和可靠性,有必要开展对双界面智能卡的侧信道分析方法的研究。本研究以提升对一类双界面商用智能卡的侧信道攻击方法的技术效果为最终目标,从减少攻击成功时所使用的能量迹、电磁迹条数的角度出发,开展两个阶段的侧信道分析研究:以多种AES算法实现为研究目标的通用数据增强方法效果量化研究、以一类双界面商用智能卡上ECDSA实现为研究目标的侧信道分析实证研究。
	
	本文的主要研究成果如下:
	
	(1)本文从数据增强的角度,研究如何提升DL-SCA技术效果。本文提出一种通用数据增强方法,该方法能够自适应地搜索数据增强策略参数,增广出新的样本用于训练深度学习模型,进而提升DL-SCA效果。本文提出的通用数据增强方法能够显著提升多种密码实现的 DL-SCA 效果。与未采用数据增强方法的DL-SCA效果相比,在AES\_HD、DPA v4、ASCADf(N=0/N=50/N=100)三种数据集五个场景下,成功条数分别降低21\%、25\%、24\%、48\%以及5\%。与采用其他数据增强方法的DL-SCA效果相比,在AES\_HD、DPA v4、ASCADf(N=0/N=50)三种数据集四个场景下,成功条数分别降低22\%、40\%、10\%以及30\%。
	
	(2)本文以一款双界面商用智能卡ECDSA实现为例,开展DL-SCA相关研究,应用\chapref{chap:search1}所提方法,有效提升了DL-SCA在双界面商用智能卡分析场景中的技术效果。在缺少双界面商用智能卡产品设计细节、ECDSA实现所使用的防护对策未知的情况下,成功实现了一款具有ECDSA签名功能的双界面商用智能卡的破解。深度学习侧信道攻击的训练阶段计算量减少到原来的$\frac{9}{50000}$,使得实施深度学习侧信道攻击变得可行。在不优化深度学习模型的情况下,将包含采集时间在内的攻击时间从理论上超过七百万年减少到4天,并通过实际攻击进行了验证,显著降低了将侧信道分析和格方法结合起来恢复ECDSA私钥的时间。本文综合选择了量化、去抖动、高斯模糊、取多个极值点、时间模版匹配等多种预处理技术及其组合,使用精确率优作为优化数据增强策略参数的指标,提出基于假设检验的敏感信息泄漏估计方案,对一款双界面商用智能卡进行攻击,在结合格方法的情况下成功恢复ECDSA的私钥。使用本文提出的基于深度学习的智能卡ECDSA实现的侧信道分析方法,可以在使用1天采集电磁迹、使用3天对电磁迹进行预处理、使用500秒进行深度学习模型的训练和预测、使用400秒进行进行格基规约的情况下,恢复ECDSA私钥。本文的分析思路和技术方法可以应用于其他类型的双界面商用智能卡分析场景中,为开展其他类似密码硬件单元的分析与破解提供可行的思路和真实的案例。
	\section{未来工作展望}
	本文从数据增强的角度,提升了DL-SCA技术效果,在多个AES实现公开数据集以及一个ECDSA实现私有数据集上完成了验证。数据增强方法具有进一步研究的价值:
	\begin{itemize}
		\item 预处理和数据增强之间的转化关系有待探究。预处理和数据增强都是对迹进行某种变换,提高深度学习侧信道攻击技术效果,但是它们的变换方式有可能完全相反。例如,数据增强向训练数据添加噪声,但是预处理力求减小噪声提高信噪比。数据增强只用于深度学习的训练数据,用于攻击的能量迹不会进行数据增强。只有用于攻击的能量迹才会进行预处理操作。探究预处理和数据增强之间的转化关系有助于同时利用预处理和数据增强的优势,突破只使用数据增强或只使用预处理带来的技术瓶颈,进一步提升DL-SCA的实用性。
		\item 本文提出数据增强方法中的数据增强单元具有良好的扩展性,其中的数据增强子策略可以根据不同的侧信道攻击需求进行替换或增加,以适应更复杂算法实现的数据集。本研究中因为所选用的三种数据增强子策略几乎正交,所以顺序可以互换\footnote{仅交换执行子策略的顺序可以通过调整子策略参数完成。}。当选择更多数据增强子策略时,子策略之间的耦合关系对DL-SCA技术效果的影响也值得探究。
		\item 本文提出的通用数据增强方法,在理论上适用于不同的深度学习模型。数据增强旨在增广训练集,此过程与其他直接优化深度神经网络的技术并不冲突。因此,可以同时应用通用数据增强方法和深度神经网络优化技术来提升DL-SCA技术效果,但是本文未进行相应实验进行验证。
	\end{itemize}
}