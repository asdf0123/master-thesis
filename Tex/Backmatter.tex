%---------------------------------------------------------------------------%
%->> Backmatter
%---------------------------------------------------------------------------%
\chapter[致谢]{致\quad 谢}\chaptermark{致\quad 谢}% syntax: \chapter[目录]{标题}\chaptermark{页眉}
%\thispagestyle{noheaderstyle}% 如果需要移除当前页的页眉
%\pagestyle{noheaderstyle}% 如果需要移除整章的页眉

“夫夷以近,则游者众;险以远,则至者少。而世之奇伟、瑰怪,非常之观,常在于险远,而人之所罕至焉,故非有志者不能至也。有志矣,不随以止也,然力不足者,亦不能至也。有志与力,而又不随以怠,至于幽暗昏惑而无物以相之,亦不能至也。然力足以至焉,于人为可讥,而在己为有悔;尽吾志也而不能至者,可以无悔矣,其孰能讥之乎?此余之所得也。”

就学业而言,本人有幸具备志、力、物,完成了硕士阶段的工作。在这这多方面的因素共同影响下,我才可能完成研究内容和独立论文并据此申请学位。

志,就是在达成目标之前憋着一口气。

力,我觉得来自于本科阶段和研究生阶段学的各方面的知识。

物,是来自周围的人的帮助。首先是父母生养。其次是老师对我工作的指导,周永彬老师、张倩老师对我的研究内容、独立论文和学位论文提供了大量帮助。邱心宽老师为独立论文的撰写提供了很多写作方面的指导。陈佳哲老师为研究内容提供了很多建议。高宜文老师对我论文撰写提供了写作方面的帮助。赵凯林老师为研究工作提供了坚实的后勤保障。然后是研究组的学长学姐和同学,冯祺学长向我传授了信息泄漏检测以及工程方面的经验,让我胜任实现信息泄露检测的工作。明经典学长、李慧中学姐为我入门侧信道分析提供了许多指导。刘月君学姐、孙硕学长、王提学长让我初步了解了格以及公钥签名方面的知识(尽管主要是通过他们讨论班的文档、论文了解),这样一来我在研究以及论文中撰写这方面内容时不至于从头摸索。金诚斌学长为我指明了一个可行的研究内容,这对于我的独立论文至关重要。乔泽华学长、姜子铭学姐在他们忙着写自己论文的时候也给我的独立论文提出了大量修改建议。姜子铭学姐的帮助没有多到足以成为作者,因此我在学位论文的致谢中再额外感谢一下。邵明尧、方磊磊是同研究组的同学也是一个宿舍的室友,他们提供了很多情绪价值。车博轩、安锦阳学弟、黄明明学长作为室友,包容了我异于常人的作息。赵禹涵学长、张泽龙学弟找我讨论过很多问题,体现了我作为一个合作者的价值。接着是小团体的一些人,唐雳雳学弟、冯驰云学弟让我深刻理解多种密码学原语,这也为我将侧信道分析从学位论文研究内容的最终目标中解耦出来,消除不必要的工作提供了很大帮助。

\rightline{2023年5月}
\chapter{作者简历及攻读学位期间发表的学术论文与其他相关学术成果}

\section*{作者简历:}
赵竟霖,中国科学院信息工程研究所在读硕士研究生

2016年9月——2020年6月,在浙江大学计算机科学与技术学院获得学士学位。

%××××年××月——××××年××月,在××大学××院(系)获得硕士学位。

2020年9月——2024年6月,在中国科学院信息工程研究所攻读硕士学位。

%工作经历:

\section*{已发表(或正式接受)的学术论文:}
{
	\setlist[enumerate]{}% restore default behavior
	\begin{enumerate}[nosep]
		%   \renewcommand{\labelenumi}{[\theenumi]}
		\item \textbf{赵竟霖},张倩*, 邱心宽, 周永彬,乔泽华。 面向基于深度学习侧信道分析的通用数据增强方法。 \textit{信息安全学报 } ,录用,CCF中文T2类期刊
		\item 张倩, 周永彬*, \textbf{赵竟霖},邱爽,刘月君,高宜文。 基于融合T表的BRAM型安全紧凑AES硬件设计。 \textit{信息安全学报 } ,录用,CCF中文T2类期刊
		
	\end{enumerate}  
} 


\section*{申请或已获得的专利:}
{
	\setlist[enumerate]{}% restore default behavior
	\begin{enumerate}
		\item 周永彬,\textbf{赵竟霖},张倩。一种基于时间模板匹配的特征点提取方法,申请号:202410584209.0
		\item 周永彬,张倩,\textbf{赵竟霖},高宜文,刘月君。一种共享核心函数的高效安全密码硬件,申请号:202210697673.1
		
	\end{enumerate}  
}  


\section*{参加的研究项目:}

{
	\setlist[enumerate]{}% restore default behavior
	\begin{enumerate}
		\item 可证明安全掩码方案设计方法及其实际安全性分析测评技术,北京市自然科学基金委员会,面上项目,4192067,2019-01至2021-12,参与
		\item 基于深度学习的侧信道分析方法研究,国家自然科学基金委员会,青年科学基金项目,62002353,2021-01至2023-12,24万元,参与
		\item 密码芯片信息泄漏深度分析与可靠防护关键技术,中华人民共和国科学技术部,国家重点研发计划,2022YFB3103804,2022-12至2025-11,参与
		\item 密码系统的信息泄漏分析方法与防护机制研究,国家自然科学基金委员会,重点项目,61632020,2017-01至2021-12,参与
		\item 融合信息泄漏的新型密码分析方法研究,国家自然科学基金委员会,通用联合重点项目,U1936209,2020-01至2023-12,参与
	\end{enumerate}  
}  
%\section*{已发表(或正式接受)的学术论文:}
%
%{
%\setlist[enumerate]{}% restore default behavior
%\begin{enumerate}[nosep]
%    \item 已发表的工作1
%    \item 已发表的工作2
%\end{enumerate}
%}

%\section*{申请或已获得的专利:}
%
%(无专利时此项不必列出)

%\section*{参加的研究项目及获奖情况:}


\cleardoublepage[plain]% 让文档总是结束于偶数页,可根据需要设定页眉页脚样式,如 [noheaderstyle]
%---------------------------------------------------------------------------%
