%---------------------------------------------------------------------------%
%->> Frontmatter
%---------------------------------------------------------------------------%
%-
%-> 生成封面
%-

\maketitle% 生成中文封面
\MAKETITLE% 生成英文封面
%-
%-> 作者声明
%-
\makedeclaration% 生成声明页
%-
%-> 中文摘要
%-
\intobmk\chapter*{摘\quad 要}% 显示在书签但不显示在目录
\setcounter{page}{1}% 开始页码
\pagenumbering{Roman}% 页码符号

%中文摘要、英文摘要、目录、论文正文、参考文献、附录、致谢、攻读学位期间发表的学术论文与其他相关学术成果等均须由另页右页(奇数页)开始。
%version 1
%双界面商用智能卡因为其灵活性和可靠性成为智能化社会中各种商用安全应用的重要技术,因此对双界面智能卡物理安全性的分析和评估具有重要的现实意义。抵御侧信道分析的能力是密码硬件单元物理安全性的一个重要方面。为了保障分析评估结果的准确性和可靠性,有必要开展对双界面智能卡的侧信道分析方法的研究。深度学习侧信道分析(DL-SCA)是一类强大的侧信道分析方法,但是在对现实中一款双界面智能卡椭圆曲线数字签名算法(ECDSA)实现进行侧信道分析时,还存在量化指标选择不合理、计算量巨大、采集电磁迹消耗的时间长、恢复私钥的理论时间长等问题。如何在可接受的攻击时间内恢复ECDSA的私钥成为侧信道分析在实际应用中所面临的非常关键的问题。本文以提升对一类双界面商用智能卡的侧信道分析方法的技术效果为最终目标,从减少攻击成功时所使用的能量迹、电磁迹条数的角度出发,开展两个阶段的侧信道分析研究:以多种AES算法实现为研究目标的通用数据增强方法效果量化研究、以一类双界面商用智能卡上ECDSA实现为研究目标的侧信道分析实证研究。本文主要贡献如下:
%
%\textbf{面向基于深度学习侧信道分析的通用数据增强方法。}本文提出了一种通用数据增强方法,它能用于对DL-SCA的训练样本进行数据增强,提高DL-SCA技术效果。该方法主要由采用模拟退火机制的主控制器、采用组合式数据增强机制的数据增强单元以及提供侧信道攻击代价反馈信号的攻击评估单元三个核心组件构成,其核心思想是采用模拟退火机制依据反馈信号自适应地选择数据增强策略参数。该方法具有通用性和有效性,在AES\_HD、DPA v4、ASCADf(N=0/N=50/N=100)三种数据集五个场景下,与未采用数据增强方法的DL-SCA相比,\chenggongtiaoshu 分别降低21\%、25\%、24\%、48\%以及5\%。
%
%\textbf{基于深度学习的智能卡ECDSA实现的侧信道分析方法。}本文将简单的信号处理方法组合使用达到高效、准确提取特征点的目的,据此构造出的数据集使得DL-SCA计算量减小为原来的0.018\%,DL-SCA变得可行。本文利用ECDSA本身的特点对深度学习的的测试阶段进行改进,采用面向基于深度学习侧信道分析的数据增强方法以及基于假设检验的估计方案使得运用格方法恢复私钥的时间减小到400秒。实际攻击时,从采集电磁迹到恢复ECDSA私钥只用4天,速度相比于改进前提高了$2^{29}$倍以上。

%version2
%智能卡的物理安全性分析是密码工程领域研究的一项重要课题。实施侧信道分析是进行物理安全性分析的重要环节,因此有必要开展对智能卡的侧信道分析方法的研究。基于深度学习侧信道分析(DL-SCA)是一类强大的侧信道分析方法,本文以提升对一类双界面商用智能卡的侧信道分析方法的技术效果为最终目标,从\jiaodu 出发,开展两个阶段的侧信道分析研究:以多种AES算法实现为研究目标的通用数据增强方法效果量化研究、以一类双界面商用智能卡上ECDSA实现为研究目标的侧信道分析实证研究。本文主要贡献如下:

%智能卡的物理安全性分析是密码工程领域研究的一项重要课题。基于深度学习侧信道分析(DL-SCA)方法具有为智能卡物理安全性分析测评实践提供新颖解决方案和精选先进技术的巨大潜能。然而,基于深度学习的侧信道分析通常需要大量数据和强大算力才能分析成功,严重地影响了这类方法的实际攻击效果。因此,本文从\jiaodu 出发,提出了一种基于深度学习的电磁分析方法,并在一类双界面商用智能卡ECDSA实现进行了实证研究。本文主要贡献如下:
%
%%实施侧信道分析是进行物理安全性分析的重要环节,因此有必要开展对智能卡的侧信道分析方法的研究。基于深度学习侧信道分析(DL-SCA)是一类强大的侧信道分析方法,本文以提升对一类双界面商用智能卡的侧信道分析方法的技术效果为最终目标,从\jiaodu 出发,开展两个阶段的侧信道分析研究:以多种AES算法实现为研究目标的通用数据增强方法效果量化研究、以一类双界面商用智能卡上ECDSA实现为研究目标的侧信道分析实证研究。本文主要贡献如下:
%
%\textbf{提出了面向基于深度学习侧信道分析的通用数据增强方法,在侧信道领域权威公开数据集上进行了实验验证。}数据增强方法可降低基于深度学习的侧信道攻击对大量数据的需求,然而已有数据增强方法参数选取通常仅针对特定算法实现运行过程中采集到的数据集。因此,本文提出了一种通用数据增强方法,该方法主要由三个核心组件构成:采用模拟退火机制的主控制器、由数据增强策略参数实例化的数据增强单元以及提供侧信道攻击代价反馈信号的攻击评估单元,其核心思想是主控制器依据反馈信号自适应地调整数据增强策略参数。本文在AES\_RD、AES\_HD、DPA v4、ASCADf(N=0/N=50/N=100)四种数据集六个场景下开展了对比研究。实验结果表明,采用本文提出的数据增强方法的DL-SCA相比于未采用数据增强方法的DL-SCA,\chenggongtiaoshu 分别减少0、21\%、25\%、24\%、48\%以及5\%,证明了本方法的通用性。该方法可适用于双界面智能卡ECDSA实现采集的数据集。
%
%%\textbf{进行了基于深度学习的针对一款双界面商用智能卡ECDSA实现的电磁分析方法研究,给出了相应参考实现。}为了解决在攻击型号为J3D081的Java卡的ECDSA私钥时,DL-SCA量化指标选择不合理、计算量巨大、采集电磁迹消耗的时间长、恢复私钥的理论时间长等问题,本文应用信号处理技术、基于假设检验的估计方案和本文提出的面向基于深度学习侧信道分析的通用数据增强方法成功实施了攻击。本文将简单的信号处理方法组合使用达到高效、准确提取特征点的目的,据此构造出的数据集使得DL-SCA计算量减小到$1.727\cdot10^{11}$浮点运算次数,为原来的0.16\%,DL-SCA变得可行。采用面向基于深度学习侧信道分析的数据增强方法以及基于假设检验的估计方案使得运用格方法恢复私钥的时间减小到400秒。实际攻击时,从采集电磁迹到恢复ECDSA私钥只用4天,速度相比于改进前提高了$2^{29}$倍以上。
%
%%version3
%\textbf{提出了一种基于深度学习的电磁分析方法,在一款双界面商用智能卡ECDSA实现上进行了实验验证,并给出了相应参考实现。}DL-SCA在实际应用时存在实施时间长、真阳性率低的问题。%为了恢复型号为J3D081的双界面商用Java卡的ECDSA私钥,本文需要将DL-SCA与格方法结合。但是ECDSA实现运行时间长导致直接现有DL-SCA难以在可接受的时间内完成、格方法对敏感信息泄漏错误敏感的特点导致即使在现有DL-SCA完成的情况下运用格方法也难以在可接受的时间内恢复私钥。
%因此,
%本文提出了一种基于深度学习的电磁分析方法,该方法包括特征点提取、模板构建以及模板匹配三个阶段。具体地,采用\yuchuli 技术并构造了用于深度学习模型的数据集,基于上述通用数据增强方法构建了适用于双界面商用智能卡ECDSA实现的深度学习模型,采用了\jiashejianyanguji 方法提高敏感中间值估计值的真阳性率。%实验结果表明,在使用十万条电磁子迹进行分析、独立进行五次进行深度学习模型的训练和测试、每次使用五千条电磁子迹进行测试、显著水平$\alpha=0.51056$的情况下,基于深度学习的电磁分析得到的敏感信息泄漏真阳性率由91.35\%提升到100\%。达到后续使用由敏感信息泄漏恢复私钥的格方法的条件。
%实验结果表明,在采集5000条电磁迹时,%【每条电磁迹拆成30条电磁子迹进行训练和测试。】
%采用基于深度学习的电磁分析方法,恢复了256比特一次性随机数(nonce)的部分比特。从中找出119条签名,这些签名对应的nonce有连续4比特正确恢复且这些比特的值为0。达到了使用由nonce部分比特恢复私钥的格方法的条件。
%%本文提出并使用{\yuchuli}的预处理技术将理论上运用格方法恢复私钥的时间从约$2\times10^{85}$年减小到约七百万年,使用面向基于深度学习侧信道分析的通用数据增强方法将理论上运用格方法恢复私钥的时间从约七百万年减少到约12.5年,提出并使用{\jiashejianyanguji}技术将理论上运用格方法恢复私钥的时间从约12.5年减少到约400秒。实际攻击中,需要1天采集电磁迹、3天进行电磁迹预处理和数据增强、1小时进行深度学习模型的训练和预测、1秒进行敏感泄漏信息估计、400秒运用格方法从敏感信息泄漏中恢复私钥,在成功恢复ECDSA私钥的前提下总时间显著减少。

智能卡的物理安全性分析是密码工程领域研究的一项重要课题。基于深度学习侧信道分析(DL-SCA)方法具有为智能卡物理安全性分析提供技术支持的能力。然而,%基于深度学习的侧信道分析通常需要大量训练数据才能分析成功
针对椭圆曲线数字签名算法(ECDSA)实现的DL-SCA还存在样本数量少、样本维度高的问题
,严重地影响了DL-SCA的实用性。因此,本文从通用数据增强和特征点精确提取两个角度出发,提出了一种基于深度学习的电磁分析方法,并在一类双界面商用智能卡ECDSA 实现进行了实证研究。本文主要贡献如下:

\textbf{提出了一种面向基于深度学习侧信道分析的通用数据增强方法,并在本领域四个公开的权威数据集上进行了实验验证。}基于深度学习的侧信道攻击方法通常需要大量训练数据,数据增强为解决这一问题提供了技术手段,然而,现有数据增强方法仅仅适用于特定算法实现运行过程中采集到的数据集。本文提出了一种通用数据增强方法,核心思想是主控制器依据反馈信号自适应地调整数据增强策略参数。该方法主要包括三个核心组件,采用模拟退火机制的主控制器、由数据增强策略参数实例化的数据增强单元以及提供侧信道攻击代价反馈信号的攻击评估单元。使用本领域四种公开数据集AES\_RD、AES\_HD、DPA v4、ASCADf(N=0/N=50/N=100)对随机延迟防护AES软件实现、无防护AES硬件实现、无防护AES软件实现、掩码(有三种随机延迟采样点数为0、50、100的场景)防护AES软件实现六种场景开展了对比实验研究。实验结果表明,与已有未采用数据增强方法的DL-SCA相比,采用该方法的DL-SCA成功开展攻击所需要的能量迹/电磁迹的条数分别减少0、21\%、25\%、24\%、48\%以及5\%。实验结果验证了该方法的有效性。

\textbf{基于上述通用数据增强方法,提出了一种针对ECDSA实现的基于深度学习的电磁分析方法,并以一款双界面商用智能卡为示例对象,对该方法进行了实验验证。}对于目标双界面智能卡上多次ECDSA签名操作,如果有64条签名使用的nonce泄漏的连续4比特0被准确恢复,那么使用格方法可以恢复ECDSA私钥。本文方法包括特征点提取、模板构建以及模板匹配三个阶段。具体地,采用基于时间模板匹配的特征点提取技术并构造用于深度学习模型的数据集,基于本文提出的数据增强方法构建适用于双界面商用智能卡ECDSA实现的深度学习模型,采用了基于假设检验的模板匹配方法提高\zyx。利用被恢复的比特泄漏可以过滤出119个含有连续4比特0的nonce,达到了使用由nonce部分比特恢复私钥的格方法的条件。

\keywords{侧信道分析,智能卡攻击,深度学习,数据增强}% 中文关键词
%\keywords{中国科学院大学,学位论文,模板}% 中文关键词
%-
%-> 英文摘要
%-
\intobmk\chapter*{Abstract}% 显示在书签但不显示在目录

%Chinese abstracts, English abstracts, table of contents, the main contents, references, appendix, acknowledgments, author's resume and academic papers published during the degree study and other relevant academic achievements must start with another right page (odd-numbered page).

The analysis of the physical security of smart cards is an important topic in the field of cryptographic engineering. The deep learning side-channel analysis (DL-SCA) method provides technical support for the physical security analysis of smart cards. %However, deep learning-based side-channel analysis typically requires a large amount of training data to be successful, seriously affecting its practicality.
However, DL-SCA targeting the Elliptic Curve Digital Signature Algorithm (ECDSA) is still faced with issues of a small number of samples and high sample dimensions, severely impacting its practicality. 
Therefore, the thesis introduces an electromagnetic analysis method based on deep learning from the perspectives of general data augmentation and precise point of interest extraction, and conducts empirical research on ECDSA implementation of a class of dual-interface commercial smart cards. The main contributions of the thesis are as follows:

\textbf{A general data augmentation method for DL-SCA is proposed, and experimental verification on four authoritative public datasets in this field is conducted.} Deep learning-based side-channel attack methods typically require a large amount of training data, and data augmentation provides a technical means to address this issue. However, existing data augmentation methods are only applicable to datasets collected during the execution of specific algorithm implementations. The thesis introduces a general data augmentation method, with the core idea of the main controller adaptively adjusting data augmentation strategy parameters based on feedback signals. This method mainly consists of three core components: a main controller using simulated annealing mechanism, data augmentation units instantiated by data augmentation strategy parameters, and an attack evaluation unit providing side-channel attack cost feedback signals. Comparative experimental studies were carried out on six scenarios, including randomized delay protection AES software implementation, unprotected AES hardware implementation, unprotected AES software implementation, and masking protection AES software implementation with three different scenarios of random delay sample points (N=0/N=50/N=100), using four public datasets in this field: AES\_RD, AES\_HD, DPA v4, and ASCADf. The experimental results show that compared to DL-SCA without data augmentation, the proposed method reduces the measurements to disclose (MTD) by 0\%, 21\%, 25\%, 24\%, 48\%, and 5\% in the six scenarios. The experimental results validate the effectiveness of this method.

%\textbf{The thesis presents a deep learning-based electromagnetic analysis method and uses a dual-interface commercial smart card ECDSA implementation as the example object to validate this method.} 
\textbf{Based on deep learning and the mentioned general data augmentation method, an electromagnetic analysis method targeting ECDSA implementations is proposed, and a dual-interface commercial smart card is used for validation.} 
For multiple ECDSA signature operations on the target dual-interface smart card, if 4 consecutive bits of 0 in the nonce used in 64 signatures are accurately recovered, the ECDSA private key can be reconstructed using the lattice-based method. The electromagnetic analysis method in this thsis consists of three stages: points of interest extraction, template construction, and template matching. Specifically, a point of interest extraction technique based on time template matching is used to construct a dataset for the deep learning model, suitable for the dual-interface commercial smart card ECDSA implementation. The data augmentation method proposed in this thsis is employed to enhance the accuracy of recovered nonce bits of 0 using a template matching method based on hypothesis testing. By utilizing the recovered bit leakage, 119 nonces containing 4 consecutive bits of 0 are filtered out, meeting the conditions for using the lattice method to recover the private key using partially recovered nonce bits.
{
%Dual-interface commercial smart cards have become an important technology for various security applications in the smart society due to their flexibility and reliability, making the analysis and evaluation of their physical security of great practical significance. The ability to resist side-channel analysis is an important aspect of the physical security of cryptographic devices. To ensure the accuracy and reliability of analysis and evaluation results, it is necessary to conduct research on side-channel analysis methods for dual-interface smart cards. Deep-learning based side-channel analysis (DL-SCA) is a powerful type of side-channel analysis method, but the way to implement DL-SCA with existing computing resources, existing storage space, and acceptable attack time is a crucial issue faced in practical applications. This research aims to improve the technical effectiveness of side-channel analysis methods for a certain type of dual-interface commercial smart cards. From the perspective of reducing the time for acquisition, DL-SCA and the lattice method, this study conducts two side-channel analysis research: a quantitative study of the efficacy of general data augmentation methods targeting various AES algorithm implementations, and an empirical study of side-channel analysis targeting the ECDSA implementation on a certain type of dual-interface smart card. The main contributions of the thesis are as follows:

%\textbf{The thesis proposed a general data augmentation method for side-channel analysis.} In order to address the issue that existing data augmentation methods often heavily rely on expert knowledge in parameter selection and are only suitable for data augmentation of specific algorithm implementation datasets, a new data augmentation method is proposed in this thesis. The key of this method is adaptively selecting data augmentation strategy parameters based on simulated annealing mechanism according to feedback signals. A comparative study is conducted in six scenarios using four authoritative public datasets in the side-channel fields of AES\_RD, AES\_HD, DPA v4, and ASCADf (N=0/N=50/N=100). Experimental results show that DL-SCA with the proposed data augmentation method reduces measurements to disclose (MTD) by 0\%, 21\%, 25\%, 24\%, 48\%, and 5\% in the six scenarios when compared with DL-SCA without data augmentation, demonstrating the effectiveness of this method.

%\textbf{The thesis conducted a study on the electromagnetic analysis method based on deep learning for the ECDSA implementation of a dual-interface commercial smart card and provided corresponding implementations.} In order to address issues such as unreasonable selection of DL-SCA quantification indicators, huge computational complexity, long time consumption for electromagnetic trace acquisition, and long theoretical time for recovering the ECDSA private key when attacking the ECDSA private key of a Java card model J3D081, this thesis successfully carried out the attack using signal processing techniques, estimation schemes based on hypothesis testing, and the proposed general data augmentation method for DL-SCA. The combination of simple signal processing methods is used to efficiently and accurately extract points of interest, and the constructed dataset reduces the computational complexity of DL-SCA to $1.727 \times 10^{11}$ floating point operations, only 0.16\% of the original amount, making DL-SCA feasible. The use of data augmentation methods for deep learning-based side-channel analysis and estimation schemes based on hypothesis testing reduces the time for key recovery to 400 seconds using the lattice method. During the actual attack, the time from electromagnetic trace acquisition to ECDSA private key recovery is only four days, with a speed improvement of over $2^{29}$ times compared to before the enhancement.
    %- the current style, comment all the lines in plain style definition.
    
%\textbf{The thesis proposed a deep learning-based electromagnetic analysis method, conducted experimental verification on a commercial dual-interface smart card based ECDSA implementation and gave out corresponding reference implementation.} A deep learning-based electromagnetic analysis method is proposed in this study, and experimental verification is conducted on a commercial dual-interface smart card implementing ECDSA, with corresponding reference implementation provided. To recover the ECDSA private key of a Java card model J3D081, the integration of DL-SCA with lattice methods is necessary. However, the extended runtime of the ECDSA implementation makes it difficult for the existing DL-SCA to complete within an acceptable time frame. Furthermore, the character that the lattice methods are sensitive to errors of estimated intermediate value leakage complicates the timely recovery of the private key even after successful DL-SCA execution. In this research, a two-stage points of interest extraction preprocessing technique is proposed and utilized to reduce the theoretical time required to recover the private key using the lattice method from approximately $2 \times 10^{85}$ years to around seven million years. Additionally, a generic data augmentation method for DL-SCA is applied to reduce the theoretical recovery time using the lattice method from about seven million years to approximately 12.5 years. A hypothesis testing-based intermediate value leakage estimation technique is introduced to further reduce the theoretical recovery time to around 400 seconds. In practical attacks, one day is required for electromagnetic trace collection, three days for preprocessing and data augmentation of electromagnetic traces, one hour for training and prediction of the deep learning model, one second for intermediate value leakage estimation, and 400 seconds for applying the lattice method to recover the private key from the leaked intermediate value, resulting in a significant reduction in total time under the premise of successfully recovering the ECDSA private key.
}

%version 4
%The analysis of the physical security of smart cards is an important topic in the field of cryptographic engineering. The deep learning based side-channel analysis (DL-SCA) method has the potential to provide novel solutions and select advanced technologies for the evaluation and practice of physical security analysis of smart cards. However, DL-SCA typically requires a large amount of data and powerful computing power to be successful, seriously affecting its practical attack effectiveness. Therefore, the thesis proposes an electromagnetic analysis method based on deep learning from the perspectives of general data augmentation and precise point of interest (POI) extraction, and conducts empirical research on a type of dual-interface commercial smart card with an elliptic curve digital signature algorithm (ECDSA) implementation. The main contributions of the thesis are as follows:
%
%\textbf{The thesis proposes a general data augmentation method for DL-SCA and conducts experiments on authoritative public datasets in the field of side-channel analysis.} The data augmentation method can reduce the demand for a large amount of data in DL-SCA. However, current data augmentation methods do well for datasets of specific algorithm implementations. Therefore, the thesis proposes a general data augmentation method, which is mainly composed of three core components: a main controller using a simulated annealing algorithm, data augmentation units instantiated by data augmentation strategy parameters, and an attack evaluation unit providing feedback signals on side-channel attack costs. The core idea is that the main controller adaptively adjusts data augmentation strategy parameters based on feedback signals. This paper conducts comparative studies in six scenarios on four datasets (AES\_RD, AES\_HD, DPA v4, ASCADf (N=0/N=50/N=100)). The experimental results show that the DL-SCA with the data augmentation method proposed in the thesis reduces measurements to disclose (MTD) by 0\%, 21\%, 25\%, 24\%, 48\%, and 5\%, respectively, compared to DL-SCA without data augmentation, demonstrating the universality of this method. This method can be applied to datasets collected from dual-interface smart card with ECDSA implementation.
%
%\textbf{The thesis proposes a deep learning-based electromagnetic analysis method, conducting experiments on a dual-interface commercial smart card with an ECDSA implementation and providing corresponding reference implementation.} DL-SCA is faced with challenges like long implementation times and low true positive rates in practical applications. Therefore, the thesis introduces a deep learning based electromagnetic analysis method, which includes three stages: POI extraction, template construction, and template matching. Specifically, it adopts a POI extraction technique based on time template matching and constructs a dataset for deep learning models. Utilizing the general data augmentation method mentioned above, a deep learning model suitable for dual-interface commercial smart card with an ECDSA implementation is built. A template matching method based on hypothesis testing is used to improve the true positive rate of estimated sensitive intermediate values. The experimental results show that, when collecting 5000 electromagnetic traces, using the deep learning-based electromagnetic analysis method successfully recovers some bits of 256-bit nonces. 119 signatures are filtered, with continuous 4 correctly recovered bits corresponding to the nonce, all having a value of 0. This meets the conditions for using the lattice method to retrieve ECDSA private key.

\KEYWORDS{Side-channel Analysis, Attack on Smart Card, Deep Learning, Data Augmentation}% 英文关键词
%\KEYWORDS{University of Chinese Academy of Sciences, Thesis, LaTeX Template}% 英文关键词

\pagestyle{enfrontmatterstyle}%
\cleardoublepage\pagestyle{frontmatterstyle}%

%---------------------------------------------------------------------------%
