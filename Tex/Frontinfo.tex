%---------------------------------------------------------------------------%
%->> Titlepage information
%---------------------------------------------------------------------------%
%-
%-> 中文封面信息
%-
\confidential{}% 密级:涉密论文或延迟公开论文填写
\schoollogo[scale=0.095]{ucas_logo}% 校徽
\title{面向一类双界面商用智能卡的侧信道分析方法研究}% 论文中文题目
\author{赵竟霖}% 论文作者
%\advisor{周永彬~教授\\中国科学院信息工程研究所}% 指导教师:姓名 专业技术职务 工作单位
\advisor{周永彬~教授\\南京理工大学}% 指导教师:姓名 专业技术职务 工作单位
%\advisor{指导教师一\\指导教师二\\指导教师三}% 多行指导教师示例
\degree{硕士}% 学位:学士、硕士、博士
\degreetype{工学}% 学位类别:理学、工学、工程、医学等
\major{网络空间安全}% 一级/二级学科专业名称,领域名称需要与学籍信息一致
\institute{中国科学院信息工程研究所}% 院系名称
%\institute{中国科学院力学研究所\\流固耦合实验室}% 多行院系名称示例
\date{2024~年~6~月}% 毕业日期:夏季为6月、冬季为12月
%-
%-> 英文封面信息
%-
\TITLE{Side-Channel Analysis against a Type of Dual-Interface Smart Cards}% 论文英文题目
%\TITLE{LaTeX Thesis Template of UCAS}% 论文英文题目
\AUTHOR{ZHAO Jinglin}% 论文作者
\ADVISOR{Supervisor: Professor ZHOU Yongbin}% 指导教师
\DEGREE{Master}% 学位:Bachelor, Master, Doctor, Postdoctor。封面据英文学位名称自动切换,需确保拼写准确
\DEGREETYPE{Science in Engineering}% 学位类别:Philosophy, Natural Science, Engineering, Economics, Agriculture 等
\MAJOR{Cyberspace Security}% 二级学科专业名称
\INSTITUTE{Institute of Information Engineering, Chinese Academy of Sciences}% 院系名称
\DATE{June, 2024}% 毕业日期:夏季为June、冬季为December
%---------------------------------------------------------------------------%
