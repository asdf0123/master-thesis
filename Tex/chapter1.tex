\chapter{绪论}\label{chap:introduction}{
	\section{研究背景}
	双界面智能卡具有开阔的发展前景以及广泛的应用环境。因此,它的密码安全性成为学术界与产业界关注的焦点。
	
	在2022年,全球智能卡市场已达到142.3亿美元的规模。而根据预测,自2023年起至2030年,市场的复合年增长率将达到5.7\%\citep{GVR-1-68038-464-2}。预计到2030年时,智能卡市场的规模将扩展至217.3亿美元。这种明确的趋势反映出,人类对于同时兼备安全性和便捷性的支付系统的需求在逐年升温,而对于高效强大的身份识别和认证方式的追求也日益增强。在这背景下,作为门禁控制、安全登陆程序以及数字签名等各类重要工具的智能卡,正在全球范围内发挥着其重要角色,广泛应用于身份识别和认证领域。随着各行各业对身份验证措施的要求日趋严格,我们预期智能卡的需求将出现大幅度增长。
	
	双界面智能卡是一种带有微处理器和存储器等微型集成电路芯片的、具有标准规格的功能强大且广泛使用的智能卡。它有金属接触和射频接口,兼具接触式卡和非接触式卡的通信功能,可以与多种终端设备进行交互。双界面智能卡的广泛应用范围涵盖了多个关键行业,包括金融、交通、物流、电子支付等领域。在金融领域,双界面智能卡可用于银行卡、信用卡和电子钱包等支付工具。它能够与销售终端(Point of Sales, POS)和自动取款机甚至移动电话进行交互,实现安全的支付和身份验证功能。在交通和物流领域,双界面智能卡被广泛应用于门禁系统、地铁票务系统、高速公路收费系统等。它能够快速、便捷地进行身份验证和票务管理,提高交通运输的效率和安全性。此外,双界面智能卡还在电子身份证、电子护照、企业门禁控制和无线通信等领域发挥着重要作用。它的灵活性和可靠性使其成为智能化社会中各种安全应用的首选技术。总的来说,双界面智能卡凭借其广泛的应用领域和卓越的安全性能,成为了当今智能卡技术的重要组成部分,为各行各业的安全交互提供了可靠的解决方案。
	
	尽管算法在理论上足够安全,但是在实际密码硬件单元中密码算法实现还可能受到物理分析的威胁。物理分析包括侧信道分析、故障注入分析\citep{Biham97}及逆向工程\citep{Perrin17}等多种手段。在众多的物理分析方法中,侧信道分析因其不需要干预程序的正常执行而具有较大的危害。此时系统的安全性就不仅依赖于算法和协议的安全性,还依赖于算法和协议的安全实现。实际应用中,密码芯片运行时会具有某种特定的物理特征,而这些特征会泄漏内部状态,分析人员可以利用这些信息泄漏破译密码系统内部的秘密信息,甚至会破译出密码系统内部的某些秘密算法或者密码结构,因而对密码系统的实际安全性造成极大威胁。利用这类信息泄漏的密码分析方法称为侧信道分析(Side-Channel Analysis)或侧信道攻击(Side-Channel Attack),而被利用的信息泄漏被称为侧信息。在密码系统的实际应用场景下,攻击者除了可以获得传统黑盒密码分析模型下可以获得的密码算法的输入与输出,还可以观测到多种有益于密码分析的侧信息泄漏。在2020年,Camurati等\citep{Camurati20}只使用无线电信号恢复使用\textit{Google Eddystone}\citep{Eddystone}协议的小型低功耗设备的AES密钥,进而通过认证。在2021年,Roche等\citep{Roche21}通过侧信道攻击可以获取谷歌安全产品\textit{Google Titan Security Key}\citep{Titan}的签名私钥进而伪造签名。
	
	侧信道攻击的威胁已经得到了重视。
	
	在政策层面,为了应对这种威胁并提高密码硬件单元的安全性,国内外的主流标准化组织和密码测评部门提出了具体的安全性要求和侧信道分析流程。例如,美国的FIPS 140-3标准\citep{FIPS140-3}和ISO/IEC 19790\citep{ISO/IEC19790}国际标准将硬件单元在侧信道攻击下的安全性作为重要要求。而ISO/IEC 17825\citep{ISO/IEC17825}国际标准则更加详细地对密码硬件单元在侧信道攻击下的安全性以及防护能力进行了概要性等级划分。此外,国际标准Common Criteria\citep{CCMB-2017-04-001}根据物理安全性定义了EAL1-EAL7,它们分别对应七个不同的安全级别。而中国国家密码管理局制定的《安全芯片密码检测准则》GM/T 0008-2012\citep{GM/T0008}、《密码模块安全技术要求》MG/T 0028-2014\citep{GM/T0028}以及《密码模块非入侵式攻击缓解技术指南》GM/T 0083-2020\citep{GM/T0083}则明确规定了密码模块必须具备针对地域典型的侧信道攻击(如能量分析攻击、电磁分析攻击和计时攻击等)的安全防护能力。以上标准和准则的出台,为增强密码硬件系统抵御侧信道攻击的能力提供了明确的指导,并对密码硬件单元的安全性进行了全面考量。
	
	在商业领域,知名芯片公司纷纷将增强抵御侧信道攻击的能力作为确保产品物理安全性的重要手段。例如,2018年,Intel主动公开了预测执行侧信道缓存计时漏洞L1终端故障(L1 Terminal Fault,L1TF),并同时公布了完整的防御措施\citep{L1TF}。为抵御类似风险,Intel还在硬件层面进行了相应改进\citep{AffectedProcessors}。2019年,Xilinx加强了其Zynq UltraScale+系列产品的抵御侧信道攻击能力,该产品集成了RSA认证加密机制和密钥刷新方案以抵御差分能量攻击和选择明文攻击\citep{UG1085}。AMD在2023年更新CPU微代码以防范了通过侧信道泄漏特权信息的Inception漏洞\citep{AMD-SB-7005}。
	
	可以看出,在国内外主流标准化组织、密码测评部门以及芯片公司中,抵御侧信道攻击的能力被视为评估密码硬件单元安全性的重要指标之一。进行侧信道安全检测的主要方法为“攻击驱动”,需要系统地尝试各种侧信道分析方法,以判断其能否对密码产品的安全性构成威胁\citep{ISO/IEC15408, GB/T18336}。确保密码硬件单元物理安全性检测评估结果的准确性和可靠性,高度依赖于所使用的先进、高效的侧信道分析方法。因此,为确保密码硬件单元物理安全性评估体系的准确性、可靠性和前沿性,有必要持续深入研究并发展先进、高效的侧信道分析方法。这将有助于改善和提升密码硬件单元信息泄露分析和物理安全性测评的技术能力,并对建立先进的密码硬件单元物理安全性评估体系具有重要意义。
	
	在实际应用中,侧信道攻击的效果受多种因素制约。能量迹、电磁迹是侧信道攻击流程中的关键要素,因此它们的数量能对侧信道攻击的效果产中严重影响。在能量迹、电磁迹充足的情况下,如分析AES\_RD\citep{AESRD}、AES\_HD\citep{AESHD}、DPA v4\citep{dpav4}、ASCAD\citep{ASCAD}等公开数据集时,分析人员可以使用充足的数据较为准确地刻画泄漏模型,进而完成设备密钥的恢复。但是在一些重要的密码应用场景中,设计人员通常使用多种防护方案对攻击者获得侧信道攻击的样本进行限制。如VMware为了提高ESXi主机安全性会激活锁定模式,在连续多次验证失败后,智能卡上的芯片将锁定。如果智能卡已锁定,则只有选定人员才能将其解锁\citep{GUID-04636353-4A11-4874-9D59-7F4F4E5CF1F}。Zynq UltraScale+的滚动密钥(Key Rolling)明确限定了密钥的使用次数\citep{UG1085}。除此之外,攻击者获取能量迹、电磁迹的速度还受算法实现执行速度的限制,如针对智能卡上RSA-1024签名过程,需要约1s才能采集一条能量迹。在这种条件下,攻击者很难获取并使用大量数据刻画泄漏模型。在样本不足的情况下,侧信道分析的技术效果会因为建模数据量受限而严重下降,进而无法成功恢复密钥。因此,探讨样本数量有限情况下的提升侧信道攻击效果的方法,具有重要的现实意义。本文以提升对一类双界面商用智能卡的侧信道攻击方法的技术效果为最终目标,从样本数量有限的角度出发,开展两个阶段的侧信道分析研究:以多种AES算法实现为研究目标的自适应数据增强方法效果量化研究、以一类双界面商用智能卡上ECDSA实现为研究目标的侧信道分析实证研究。
	
	\$\{研究内容1\}
	%深度学习在许多时间序列分析任务中表现出色。深度神经网络的卓越性能在很大程度上依赖于大量的训练数据,以避免过拟合。然而,现实世界中许多时间序列应用的标注数据可能是受限的,如采样密码设备获得的能量迹因为多次验证失败后加密模块会被锁定、采集时间受限等原因限制加密算法执行次数。作为提高训练数据规模和质量的有效方法,数据增强对成功应用深度学习模型到时间序列数据上显得至关重要。
	
	\$\{研究内容2\}
	\section{国内外研究现状}
	
	%深度学习(Deep Learning, DL)使用深度神经网络,通过已有的信息或者知识来对未知事物进行预测。随着硬件算力的提升和可用数据的增加,深度学习技术近年来得到了迅速发展。在诸如图像分类、目标检测、语音识别、自然语言处理等领域,深度学习技术已经取得了远超统计和人工特征提取等传统方法的性能。
	\subsection{信息泄漏预处理方法}
	将对齐和提取POI、降噪,算作ECDSA芯片卡预处理的背景、现状。
	\subsection{信息泄漏分析方法}
	TA、DL-SCA和DA。算AES和ECDSA攻击的背景、现状。
	\section{本文研究内容}
	\subsection{面向侧信道分析的自适应数据增强方法}
	问题困难性所在。解决问题的效果。
	\subsection{基于深度学习的ECDSA签名智能卡的侧信道分析方法}
	问题困难性所在。解决问题的效果。
	\section{本文组织结构}
}