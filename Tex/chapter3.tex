\chapter{面向侧信道分析的自适应数据增强方法}\label{chap:search1}{
	研究的动机。
	\section{自适应数据增强方法框架}
	\subsection{主控制器}
	\subsection{数据增强单元}
	
	\begin{algorithm}
		\caption{位移变换 shift\_deformation}\label{alg:shift}
		\begin{algorithmic}[1]
			\Statex \textbf{输入:} 能量迹$\vec x=\begin{bmatrix}
			x_1&x_2&\cdots&x_{M-1}
			\end{bmatrix}$,最大位移幅度$m_1, 0\le m_1<M$
			\Statex \textbf{输出:}能量迹$\vec y=\begin{bmatrix}
			y_1&y_2&\cdots&y_{M-1}
			\end{bmatrix}$
			\State $s\gets\{-m_1,-m_1+1,-m_1+2,\dots,m_1-1,m_1\}$
			\For{$n = 0, \dots, M-1$}
			\State $y_i:= x_{(i+s)\mod M}$
			\EndFor
		\end{algorithmic}
	\end{algorithm}
	
	\begin{algorithm}
		\caption{添加噪声 add\_noise}\label{alg:noise}
		\begin{algorithmic}[1]
			\Statex \textbf{输入:} 能量迹$\vec x=\begin{bmatrix}
			x_1&x_2&\cdots&x_{M-1}
			\end{bmatrix}$,噪声方差$m_2$
			\Statex \textbf{输出:}能量迹$\vec y=\begin{bmatrix}
			y_1&y_2&\cdots&y_{M-1}
			\end{bmatrix}$
			\For{$n = 0, \dots, M-1$}
			\State $e_i\gets N(0,m_2)$
			\State $y_i:= x_i+e_i$
			\EndFor
		\end{algorithmic}
	\end{algorithm}
	
	\begin{algorithm}
		\caption{合成少数类过采样法 SMOTE}\label{alg:smote}
		\begin{algorithmic}[1]
			\Statex \textbf{输入:} 能量迹$\vec x=\begin{bmatrix}
			x_1&x_2&\cdots&x_{M-1}
			\end{bmatrix},\vec y=\begin{bmatrix}
			y_1&y_2&\cdots&y_{M-1}
			\end{bmatrix}$,参数$m_3$
			\Statex \textbf{输出:}能量迹$\vec z=\begin{bmatrix}
			z_1&z_2&\cdots&z_{M-1}
			\end{bmatrix}$
			\State $\lambda\gets B(m_3,m_3)$
			\For{$n = 0, \dots, M-1$}
			\State $z_i:=(1-\lambda)x_i+\lambda y_i$
			\EndFor
		\end{algorithmic}
	\end{algorithm}
	\subsection{攻击评估单元}
	\section{自适应数据增强方法在AES算法实现分析中的应用}
	\subsection{实验设置}
	\subsubsection{主控制器设置}
	\begin{figure}[!h]
		\begin{center}
			\begin{tikzpicture}[node distance=23pt]
			\node[draw, rounded corners,align=center]  (start)  {设置初始温度$temp$、退火速度$\gamma$、参数搜索范围};
			%\node[draw, trapezium, trapezium left angle=70, trapezium right angle=110,below=of start](geninit){随机生成初始解$p_{sd},m_{sd},p_{an},m_{an},p_{os},m_{os}$};
			\node[draw,align=center,below=of start](geninit){随机生成初始解$p_{sd},m_{sd},p_{an},m_{an},p_{os},m_{os}$作为当前解};
			\node[draw, below=of geninit,align=center]  (getcost)  {依据当前解,分别调用\\数据增强单元、DL-SCA模型、攻击评估单元\\得到攻击代价$cost$};
			\node[draw, below=of getcost,align=center]  (gennewstate)  {扰动当前解产生新解$p_{sd}^\prime,m_{sd}^\prime,p_{an}^\prime,m_{an}^\prime,p_{os}^\prime,m_{os}^\prime$};
			\node[draw, below=of gennewstate,align=center]  (getnewcost)  {依据新解,分别调用\\数据增强单元、DL-SCA模型、攻击评估单元\\得到攻击代价$newcost$};
			\node[draw, below=of getnewcost,align=center] (calcdelta)  {$\Delta:=newcost-cost$};
			\node[draw, diamond, aspect=2, below=of calcdelta,align=center]     (deltaminus)  {$\Delta<0$?};
			\node[draw, below=of deltaminus,align=center] (acceptnew)  {接受新解(更新当前解为新解)\\$p_{sd},m_{sd},p_{an},m_{an},p_{os},m_{os}$\\$:=p_{sd}^\prime,m_{sd}^\prime,p_{an}^\prime,m_{an}^\prime,p_{os}^\prime,m_{os}^\prime$\\$cost:=newcost$};
			\node[draw, right=50pt of deltaminus,align=center] (calcp)  {按Metropolis准则\\计算概率阈值$p_{th}=e^{\frac{-\Delta}{temp}}$\\采样$p\stackrel{\$}\gets U(0,1)$};
			\node[draw, diamond, aspect=2, below=of acceptnew,align=center]     (paccept)  {是否达到\\迭代次数?};
			\node[draw, diamond, aspect=2, align=center] at (acceptnew -|calcp)    (itend)  {$p\le p_{th}$?};
			\node[draw,align=center] at (paccept -|itend)  (reserve)  {保留当前解};
			\node[draw, left=80pt of deltaminus,align=center] (update)  {降低温度\\$temp=\gamma\cdot temp$};
			\node[draw, rounded corners, below=of paccept,align=center]  (end) {运算结束\\返回当前解(最优解)};
			%\draw (paccept)--node[below]{否} ++(-4,0) |- (reserve);
			%\path[->] 
			\draw[->] (paccept) -|node[midway, near start,below] {否} (update);
			%\draw[->] (itend) --node[midway, near start,left] {否} (reserve);
			%\draw[->] (itend) --node[midway, near start,below] {是} (acceptnew);
			\graph{
				(start) -> (geninit) -> (getcost) -> (gennewstate) -> (getnewcost) -> (calcdelta) -> (deltaminus) ->["是"left] (acceptnew) ->(paccept)->["是"] (end);
				(deltaminus) ->["否"] (calcp) ->(itend)->["否"right](reserve)->(paccept);
				(update)->[to path={|- (\tikztotarget)}](gennewstate);
				(itend)->["是"below](acceptnew)
			};
			\end{tikzpicture}
			\bicaption{\enspace 主控制器}{\enspace controller}
		\end{center}
	\end{figure}
	\subsubsection{数据增强单元设置}
	
	\begin{algorithm}
		\caption{组合式数据增强}\label{alg:combined_da}
		\begin{algorithmic}[1]
			\Statex \textbf{输入:} 能量迹$\vec x=\begin{bmatrix}
			x_1&x_2&\cdots&x_{M-1}
			\end{bmatrix},\vec y=\begin{bmatrix}
			y_1&y_2&\cdots&y_{M-1}
			\end{bmatrix}$,参数$p_1,p_2,p_3,m_1,m_2,m_3$
			\Statex \textbf{输出:}能量迹$\vec z=\begin{bmatrix}
			z_1&z_2&\cdots&z_{M-1}
			\end{bmatrix}$
			\State $p\gets U(0,1)$
			\If {$p<p_1$}
			\State $x:= \mathrm {shift\_deformation}(x,m_1)$
			\EndIf
			\State $p\gets U(0,1)$
			\If {$p<p_2$}
			\State $x:= \mathrm{add\_noise}(x,m_2)$
			\EndIf
			\State $p\gets U(0,1)$
			\If {$p<p_3$}
			\State $z:= SMOTE(x,y,m_3)$
			\EndIf
		\end{algorithmic}
	\end{algorithm}
	\subsubsection{攻击评估单元设置}
	\subsubsection{基于深度学习的侧信道攻击模型设置}
	\subsection{AES算法实现数据集}
	\subsection{参数验证}
	\subsection{自适应数据增强方法在AES算法实现分析中的实证研究}
	\subsubsection{自适应数据增强方法的技术效果}
	\subsubsection{无数据增强的技术效果}
	\subsubsection{位移变形的技术效果}
	有做实验(复现)。
	\subsubsection{添加噪声的技术效果}
	有做实验(复现)。
	\subsubsection{合成少数类过采样法的技术效果}
	有做实验(复现)。
	\subsection{讨论与分析}
	总表对比结果。
}